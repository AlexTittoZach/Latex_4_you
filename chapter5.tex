\chapter{Limitations and Future Scope}

While the implementation and evaluation of computer vision-based driver drowsiness detection systems showcase substantial progress, several limitations merit critical acknowledgement. A clear understanding of these aspects is essential for steering future research and ensuring practical system deployment.

\section{Limitations}

Despite outstanding accuracy and flexibility, existing drowsiness detection approaches face key limitations:

\begin{itemize}
    \item \textbf{Environmental Sensitivity:}  
    Detection accuracy can degrade under poor lighting, extreme facial occlusions, or rapid head movements, which can obscure landmarks and increase false positives or negatives.

    \item \textbf{Individual Variability:}  
    Fixed EAR and MAR thresholds may not generalize to all users due to differences in facial geometry, eye shape, and blinking/yawning patterns, limiting system personalization.

    \item \textbf{Limited Temporal Prediction:}  
    Most current systems react to immediate drowsiness symptoms. Predicting gradual fatigue or modeling individual circadian rhythms remains challenging.

    \item \textbf{Privacy Concerns:}  
    Continuous video monitoring raises privacy issues; drivers may feel discomfort or concern over potential misuse of captured footage.

    \item \textbf{Resource Constraints:}  
    Real-time landmark detection and machine learning inference may strain computational resources on embedded or low-cost hardware, affecting deployment feasibility in edge settings.
    
    \item \textbf{Integration Challenges:}  
    Seamless interfacing with diverse vehicle models and intelligent alert mechanisms can require significant engineering effort and standardization.
\end{itemize}

\section{Future Scope}

Promising avenues for advancing drowsiness detection technology include:

\begin{itemize}
    \item \textbf{Multimodal Fusion:}  
    Combining facial analysis with physiological sensors, audio cues, and steering behavior offers robust, redundant safety monitoring, especially in challenging environments.
    
    \item \textbf{Personalized Calibration:}  
    Adaptive models leveraging short calibration routines can fine-tune thresholds and detection logic to each driver, reducing false alarms and enhancing system trust.
    
    \item \textbf{Privacy Protection:}  
    Integrating local processing, anonymization techniques, and federated learning minimizes privacy risks and builds user confidence.
    
    \item \textbf{Edge Optimization:}  
    Streamlined algorithms and hardware acceleration (e.g., ARM NEON, GPU, or NPU support) can improve real-time performance on embedded platforms.
    
    \item \textbf{Context-Aware Prediction:}  
    Systems can integrate trip context, time-of-day, and historical fatigue data for predictive, preventive alerts rather than reactive alarms.
    
    \item \textbf{Scalable Standardization:}  
    Collaborative development of standardized APIs, datasets, and performance metrics enables fair benchmarking and broad deployment across vehicle fleets and markets.
\end{itemize}

