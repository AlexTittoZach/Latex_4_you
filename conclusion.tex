\chapter{Conclusion}

Driver drowsiness detection based on computer vision marks a significant advancement in the quest for safer transportation. By shifting focus from invasive physiological monitoring to behavioral observation, modern systems prioritize practical deployability, user comfort, and robust performance under diverse conditions. Instead of relying purely on clinical signals, these solutions harness visual cues such as eye closure (EAR), yawning frequency (MAR), and facial dynamics to infer fatigue states in real time.

Recent work demonstrates that combining classical algorithms—such as Haar cascade detection and geometric analysis—with machine learning classifiers enables accurate and timely detection of drowsiness. Open-source frameworks make these techniques accessible and affordable, allowing for rapid deployment across personal vehicles, commercial fleets, and emerging autonomous transit systems. Extensive comparative evaluations confirm that vision-based approaches not only approach the accuracy of EEG/EOG methods but also greatly enhance user acceptance through non-intrusive monitoring.

Nonetheless, certain limitations remain regarding environmental robustness (e.g., poor lighting, facial occlusions), individual variability, privacy, and computational resource needs on embedded platforms. Continued research is necessary to address these challenges, enhance system generalizability, and foster standardized evaluation metrics. 

By uniting vision-based technology, smart analytics, and careful attention to human factors, the next generation of drowsiness detection systems promises to set new standards in road safety and accident prevention.
