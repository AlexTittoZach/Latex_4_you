\chapter{Theoretical Aspects}

Driver drowsiness detection using computer vision moves beyond traditional physiological monitoring, emphasizing the analysis of observable \emph{behavioral manifestations} such as eye closure and yawning. This chapter summarizes the core mathematical foundations, algorithmic architectures, and model evaluation strategies relevant to modern facial landmark-based drowsiness systems.

\section{Foundations}

Early systems relied on EEG, EOG, and ECG signals for clinical-grade detection, but these methods demanded intrusive hardware. Modern behavioral approaches focus on face and landmark tracking to capture fatigue through non-invasive means, delivering advantages in user comfort, cost, and real-time processing.

Mathematically, behavioral systems extract features like Eye Aspect Ratio (EAR) and Mouth Aspect Ratio (MAR) to serve as reliable proxies for physiological measures:
\[
    \text{EAR} = \frac{\|p_2 - p_6\| + \|p_3 - p_5\|}{2 \|p_1 - p_4\|}
\]
\[
    \text{MAR} = \frac{\|p_{51} - p_{59}\| + \|p_{52} - p_{58}\| + \|p_{53} - p_{57}\|}{2 \|p_{49} - p_{55}\|}
\]
where \(p_i\) are facial landmark coordinates. These ratios decrease below characteristic thresholds during prolonged eye closure (EAR) and increase during yawning (MAR).

\section{Facial Landmark Detection and Processing}

Detection begins with Haar Cascade classifiers, which efficiently locate facial regions in each frame. Dlib’s 68-point landmark regression further refines the localization of key features, making geometric computation feasible. Preprocessing operations—grayscale conversion and histogram equalization—normalize pixel values for robustness under variable lighting.

For stable classification, temporal filtering such as moving averages or exponential smoothing is applied:
\[
    \text{EAR}_{\text{smooth}}(t) = \frac{1}{N} \sum_{k=0}^{N-1} \text{EAR}(t-k)
\]
Threshold logic detects drowsiness when EAR remains below a user-calibrated threshold for multiple frames:
\[
    \text{Drowsy}_{\text{eye}} = 
    \begin{cases}
    1, & \text{if } \text{EAR}(t) < \tau_{\text{EAR}} \text{ for duration } T \\
    0, & \text{otherwise}
    \end{cases}
\]
Similar logic applies for yawn detection via MAR.

\section{Machine Learning Classification}

Beyond rule-based thresholds, machine learning models—especially Support Vector Machines (SVMs)—can combine EAR, MAR, blink/yawn frequency, and other features into multidimensional vectors:
\[
    \mathbf{x} = [\text{EAR}_{\text{left}}, \text{EAR}_{\text{right}}, \text{MAR}, f_{\text{blink}}, f_{\text{yawn}}, \ldots]^T
\]
SVMs are trained to separate alert and drowsy states via hyperplane optimization, with kernel methods enabling non-linear decision boundaries for improved accuracy.

\section{Evaluation Metrics}

Performance is measured by:
\newpage
\[
    \text{Accuracy} = \frac{TP + TN}{TP + TN + FP + FN}
\]
\[
    \text{Recall} = \frac{TP}{TP + FN}
\]
\[
    \Delta t = t_\text{alert} - t_\text{onset}
\]
where TP, TN, FP, FN denote true/false positive/negatives and \(\Delta t\) quantifies detection latency.

Real-world viability hinges on achieving high accuracy and low latency in diverse conditions. Safety demands sensitive recall (few missed alerts), while alert fatigue is minimized via precision tuning.

\section{Practical and Computational Considerations}

Camera-based behavioral systems are computationally efficient: Haar Cascade and landmark regression enable real-time (15–30 fps) operation even on embedded devices. Personalized calibration further refines threshold logic for individual differences, typically conducted during system setup.

\section{Conclusion}

Modern drowsiness detection systems combine computer vision, mathematical feature extraction, and machine learning within an optimized pipeline. They deliver practical, scalable, and non-invasive safety enhancements for diverse transportation environments, outperforming invasive clinical approaches while balancing accuracy, flexibility, and user comfort.
